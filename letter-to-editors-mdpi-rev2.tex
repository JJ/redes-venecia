% LaTeX rebuttal letter example. 
% 
% Copyright 2019 Friedemann Zenke, zenkelab.org
%
% Based on examples by Dirk Eddelbuettel, Fran and others from 
% https://tex.stackexchange.com/questions/2317/latex-style-or-macro-for-detailed-response-to-referee-report
% 
% Licensed under cc by-sa 3.0 with attribution required.

\documentclass[11pt]{article}
\usepackage[utf8]{inputenc}
\usepackage{fullpage}
\usepackage{xcolor}

% import Eq and Section references from the main manuscript where needed
% \usepackage{xr}
% \externaldocument{manuscript}

% package needed for optional arguments
\usepackage{xifthen}
% define counters for reviewers and their points
\newcounter{reviewer}
\setcounter{reviewer}{0}
\newcounter{point}[reviewer]
\setcounter{point}{0}

% This refines the format of how the reviewer/point reference will appear.
\renewcommand{\thepoint}{P\,\thereviewer.\arabic{point}} 

% command declarations for reviewer points and our responses
\newcommand{\reviewersection}{\stepcounter{reviewer} \bigskip \hrule
                  \section*{Reviewer \thereviewer}}

\newenvironment{point}
   {\refstepcounter{point} \bigskip \noindent {\textbf{Reviewer~Point~\thepoint} } ---\ }
   {\par }

\newcommand{\shortpoint}[1]{\refstepcounter{point}  \bigskip \noindent 
	{\textbf{Reviewer~Point~\thepoint} } ---~#1\par }

\newenvironment{reply}
   {\medskip \noindent \begin{sf}\textbf{Reply}:\  }
   {\medskip \end{sf}}

\newcommand{\shortreply}[2][]{\medskip \noindent \begin{sf}\textbf{Reply}:\  #2
	\ifthenelse{\equal{#1}{}}{}{ \hfill \footnotesize (#1)}%
	\medskip \end{sf}}

    \newcommand{\rev}[1]{{\color{purple} #1}}

\begin{document}


\section*{Response to the reviewers}

We would like to thank the reviewer for the attention that our work has
received, and their very detailed review. We will address their concerns next

\reviewersection

\subsection{Minor points}

\begin{point}
  p. 1, lines 32-36 – the doge is not the President of the republic. His rank is Prince, but he is also “first among equals”. I would reformulate as follows: “its state was governed by a representative figure, with a rank of prince, called doge, elected for life, initially by popular acclamation, then from the 12th century by all male nobles, members of the Great Council, the sovereign body of the Republic Maranini (1927); Ruggiero (1979)”.
	\label{pt:m:1}
      \end{point}

      \begin{reply}
Fixed as suggested. Thanks for the suggestion.
\end{reply}

\begin{point}
   p. 2, lines 81-82: “and then showed that the time spent in doges decreased
   extensively after that date”. I suppose the Author means: “…that the time
   doges remained in office decreased….”
   	\label{pt:m:2}
 \end{point}

 \begin{reply}
   Done as suggested. Thanks again.
 \end{reply}

\begin{point}
p. 4, figure 1 – “blue shows marriages within the same family” – I think the Author should explain that the extended family has a number of branches (sometimes 2-3, some others up to 30 or even more). When one branch has only a female member, the usual custom is to get her married to a male from a close branch in order not to lose the heritage, nor the potential votes of family members. It might be wise to explain this somewhere in the article.
   	\label{pt:m:3}
 \end{point}

 \begin{reply}
   This was a caption to a Figure. It has been extended and explained further in
   the text, citing Raines 2013. The mention to why some of those marriages took
   place when a there was a single female in the {\em ramo} was probably out of
   scope for the paper, and I could not find a precise source for it, so I left
   it out.
 \end{reply}

\bibliographystyle{unsrt}
\bibliography{marriage}
\end{document}
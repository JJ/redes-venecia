% LaTeX rebuttal letter example. 
% 
% Copyright 2019 Friedemann Zenke, zenkelab.org
%
% Based on examples by Dirk Eddelbuettel, Fran and others from 
% https://tex.stackexchange.com/questions/2317/latex-style-or-macro-for-detailed-response-to-referee-report
% 
% Licensed under cc by-sa 3.0 with attribution required.

\documentclass[11pt]{article}
\usepackage[utf8]{inputenc}
\usepackage{fullpage}
\usepackage{xcolor}
\usepackage{xr}
\externaldocument{change-point-marriages-venice}

% import Eq and Section references from the main manuscript where needed
% \usepackage{xr}
% \externaldocument{manuscript}

% package needed for optional arguments
\usepackage{xifthen}
% define counters for reviewers and their points
\newcounter{reviewer}
\setcounter{reviewer}{4}
\newcounter{point}[reviewer]
\setcounter{point}{0}

% This refines the format of how the reviewer/point reference will appear.
\renewcommand{\thepoint}{P\,\thereviewer.\arabic{point}} 

% command declarations for reviewer points and our responses
\newcommand{\reviewersection}{\stepcounter{reviewer} \bigskip \hrule
                  \section*{Reviewer \thereviewer}}

\newenvironment{point}
   {\refstepcounter{point} \bigskip \noindent {\textbf{Reviewer~Point~\thepoint} } ---\ }
   {\par }

\newcommand{\shortpoint}[1]{\refstepcounter{point}  \bigskip \noindent 
	{\textbf{Reviewer~Point~\thepoint} } ---~#1\par }

\newenvironment{reply}
   {\medskip \noindent \begin{sf}\textbf{Reply}:\  }
   {\medskip \end{sf}}

\newcommand{\shortreply}[2][]{\medskip \noindent \begin{sf}\textbf{Reply}:\  #2
	\ifthenelse{\equal{#1}{}}{}{ \hfill \footnotesize (#1)}%
	\medskip \end{sf}}

\newcommand{\rev}[1]{{\color{purple} #1}}

\begin{document}


\section*{Response to the reviewers}

We would like to thank the editor and reviewers for the attention that our work has received. We will address their concerns next.

\begin{point}
This article applies a range of statistical techniques to the distribution
of aristocratic marriages in the Republic of Venice 1398–1797 with the aim
of discovering ‘turning points’ (‘change points’ in the statistical literature,
related to the ‘critical junctures’ of the political science literature). This
is a nice problem to tackle, and the article is acceptably written, but I am
concerned that there is no clear, motivating historical question or questions.
\end{point}

\begin{reply}

  Thanks a lot for the summary, and the suggestion for alternate terms. It is
  indeed a nice problem, we can even say it is a Venice problem (allow the pun)
  since there has been for a long time a historial discussion on social mobility
  and, the definition of the noble/ruling class in it, and what were the effects
  of different legislative changes approved by different governing bodies
  (Maggior Consiglio, Consiglio dei X, Senate). The work by Chojnacki
  \cite{10.2307/202860,second:serrata,chojnacki00} and Raines
  \cite{raines2013rameau,raines2003cooptazione} has repeatedly provided data
  from several sources, from the legislative to data in legal registries or
  descriptions and letters. So far, however, an analysis of one of the
  institutions of that class, marriage, had not been performed using hard
  statistical tools.

  There is also another historical question that this paper could contribute to
  solve, or at least add more mathematically precise results that could be taken
  into account. When did the Republic of Venice reach a point when its demise
  was inevitable? A secondary question is if we can answer that question from a
  partial, if important, set of data: a time series of noble matrimonies. We
  have managed to answer first the latter (it is representative, maybe more so
  than other data, since marriage is a crucible in the lifecycle of the most
  important agent in Venetian society and politics: the {\em casata} or extended
  family), and then to the former, finding the second half of the 17th century,
  with the confluence of the loss of Crete and the start of the invasion of the
  Morea, to be such a point of no return. Of course, that is not a cause, but a
  critical juncture where different trends converged, and that is revealed by
  the mathematical analysis.

  The conclusion to the section on analysis of the different detected periods
  has been changed now to reflect this result with more precision than before.
\end{reply}


\begin{point}
Above all I am unconvinced by the methodology of the search for change
points, or of the discovered change points’ significance, either statistically
or historically.
\end{point}

\begin{reply}
The statistical significance of the points detected by the different methods has
been statistically established, and now shown in new Table \ref{tab:change.point:number}. Not only are they
significant for all three methods (Lanzante, Pettitt and SMH), but also their
p-value is essentially 0 (actually, $10^{-37}$).

From the historical point of view, we should clarify that the causes of change
points will not happen in the same year it has been statistically detected. In
manyu cases, they will have happened years and even decades before, or be the
result of a historical trend that started many years back (for instance, a
demographic or economic trend). It is clear that one of the historical events
that caused the irrelevancy of the Republic of Venice was the loss of monopoly
in the spice trade after Portugal established maritime trade routes to India,
and general irrelevancy of the Mediterranean in the geopolitical context after
the discovery of America and establishment of trade routes with them. These
events will start a cause-effect chain, but it will not be immediately felt;
some of it will produce a watershed, irreversible change point many years
later. At any rate, Section 5 on detecting change points has been completed now
with a series of subsections that analyze the most important historical change
points affecting the statistically detected ones. Please check new subsections
5.1 to 5.5 (and conclusion in section 5.7).
\end{reply}

\begin{point}
Generally, with a data set, one should begin with an examination of appropriate statistical visualizations by eye. In this case, I would ask the reader
to begin with Figure 3a (where I number Figure 3’s four panels (a)–(d), top–
bottom). Clearly there are overall trends, and these are quite reasonably
discovered in Fig.3c, once the seasonal element so obviously present in Fig.1
has been removed in Fig.3b. The residuals are shown in Fig.3d, and are not
stationary (although they may be multiplicatively so, and I think I don’t
think this is significant – rather the residuals are probably to be regarded
as structureless ‘noise’ impeding further analysis). Much the same is seen
in Fig.5. This is all fine, and leads to the conclusion obvious to the naked
eye: that there was a peak in the mid-16C (p6/178).
\end{point}

\begin{reply}
  That is roughly correct. Our intention with applying trend analysis was,
  besides finding obvious periods (which would be related to matrimonial custom)
  to reveal whether the peak seen in plain sight was actually the real one. This
  allowed us to reach a conclusion on the actual peak, which can then be
  explained (and applied in the analysis of different periods).

  Besides, the change point brought by a change of trend was so obvious that we
  couldn't help but include it in this statistical analysis, together with the
  analysis of other change points
\end{reply}  

\begin{point}
The point about number of marriages varying with total population,
which also has a 16C peak, is made on p7 but not properly explored. I
wonder if the peak would disappear if instead the plot was of marriages
divided by total population?
\end{point}

\begin{reply}
  It certainly could. That way we could differentiate what are {\em internal}
  changes in the number of marriages (which could correspond to {\em internal}
  causes related to social changes) to changes in {\em how many} nobles actually
  married among the total. The issue is that we do not have that data;
  a complete time series of population is not available, although a new
  reference added \cite{raines2013rameau,davis1962decline} do a rather good job
  of publishing certain data points in the number of nobles as well as the total
  population of the city.

  In absence of that, however, we can still perform a valid analysis, that needs
  forcefully to be combined with a historical analysis and a trend analysis on
  the number of marriages, their diversity (measured by entropy), as well as the
  number of marriages to non-nobles, a data point which certainly points more to
  {\em internal} noble marriage dynamics than to any kind of external
  influence. To this we have added in this new version the number of
  intra-family marriages (new Figure 8), another piece of data that can reveal
  real causes of the peak.

  At any rate, how marriages vary with total population is now analyzed in the
  historial context in the new description of every period in subsections of
  Section 5.
\end{reply}


\begin{point}
Change point analysis (CPA) is the attempt to seek moments at which
the statistical distribution changes, with distinct ‘before’ and ‘after’ regimes.
This makes less sense when, as here, the sense is more of continuous trends
rather than distinct regimes. A ‘change of trend’ (i.e. in the gradient of the
data) is not the best way to see this – it imputes a linear slope up, then a
change to a linear slope down; and this is not usually how maxima appear
(unless there were a sharp event causing the change. We don’t see this here).
15. The crux of the author’s CPA method is in p8/231, and it is simply
to look at differences in averages. I find 1654 unconvincing as a putative
change point in the light of this. One can see how the methods arrive at it:
it is the year at which the averages before and after differ most. Yet this
really is not a correct or useful way to describe these data; that there is a
CP in 1654 it is not a meaningful conclusion.
\end{point}

\begin{reply}
In order to reach that conclusion, we have used three established methods of
change point analysis on series of number of marriages, Lanzante's method on the
entropy series (which has been computed from the dataset independently), and
finally a non-parametric multi-change point analysis method included in the {\sf
  ecp} package, which considers all three variables at the same time. The main
change point reached by {\em all} methods is in the same period, between 1645
and 1654, with a high degree of statistical coincidence. That is not a
conclusion, neither we pretend it to be. It is a result that will help us
establish the main point of our paper, which are the historical facts that led
to this pivotal moment, or maybe we should say epoch.

In this sense, we have added an additional changepoint analysis algorithm,
\cite{killick}, to the paper; this algorithm analyzes not only the average, but
also the standard deviation, and tries to find the point that maximizes
difference in both (not only average, as before). The result is also
represented in new Figure \ref{fig:histories2.changepoint.meanvar}.
The changepoint this method yields is in in the same general area, second
half of the 17th century. We can agree that a single changepoint year, in this
case 1654, might not be a satisfactory result from the point of view of
statistics. However, using different algorithms that reach, if not the exact
same conclusion, results that are in the same area, allows us to conclude that
there is a pivotal change in the history of Venice, as reflected in noble
marriages, that caused a changepoint in the second half of the 17th century.

We can summarize by saying that while a single changepoint, computed via a
single changepoint method, might not be significant, using a number of methods
that yield results that are mathematically (and historycally) close will help us
reach a conclusion.
\end{reply}

6. Looking at the entropy (perhaps better introduced for non-experts as
‘diversity’ of marriage families, p13/377) is a nice idea. But again, look at
Fig.6. What is happening here? Are there sudden moments of change? If
one simply looks at a point separating different averages, then 1645 may
be reasonable. But this is clearly not a moment of sudden change in Fig.6.
Rather I suspect that the smoothed trend has just one moment of, if not a
sudden change, a steep gradient, in the early part of the 16C. It seems to
me that the most interesting period that is emerging here is the early 16C,
with marriages approaching a peak, but from a declining number of families.
Does that connect with historical scholarship?
7. We then move on to allow multiple change points, and the separation of
the Republic into periods. I don’t think the CPs are convincing (just look
at Fig.7), and to make them so there would require much more integration
of the statistics with the historical scholarship.
8. So in the end I’m unconvinced by the claim of ‘rigorous’ (p13/384) sta-
tistical analysis. Precise in its computations, certainly, but without enough
integrative thought or ‘letting the data speak’.
9. There is a small number of typos, misprints or faults of usage which I
will not list.
10. The references to historical scholarship on the Republic of Venice are
generally good, but (as per above) I don’t feel there is enough integration of
this with the data analysis and its meaning. It might be worth citing Wawro
and Katznelson’s recent work advocating Bayesian CPA for historical social
science.

\begin{point}
	\label{pt:1:1}
      \end{point}

      \begin{reply}
        \end{reply}

\bibliographystyle{unsrt}
\bibliography{marriage,venice,change-point}
\end{document}
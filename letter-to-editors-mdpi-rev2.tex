% LaTeX rebuttal letter example. 
% 
% Copyright 2019 Friedemann Zenke, zenkelab.org
%
% Based on examples by Dirk Eddelbuettel, Fran and others from 
% https://tex.stackexchange.com/questions/2317/latex-style-or-macro-for-detailed-response-to-referee-report
% 
% Licensed under cc by-sa 3.0 with attribution required.

\documentclass[11pt]{article}
\usepackage[utf8]{inputenc}
\usepackage{fullpage}
\usepackage{xcolor}

% import Eq and Section references from the main manuscript where needed
% \usepackage{xr}
% \externaldocument{manuscript}

% package needed for optional arguments
\usepackage{xifthen}
% define counters for reviewers and their points
\newcounter{reviewer}
\setcounter{reviewer}{0}
\newcounter{point}[reviewer]
\setcounter{point}{0}

% This refines the format of how the reviewer/point reference will appear.
\renewcommand{\thepoint}{P\,\thereviewer.\arabic{point}} 

% command declarations for reviewer points and our responses
\newcommand{\reviewersection}{\stepcounter{reviewer} \bigskip \hrule
                  \section*{Reviewer \thereviewer}}

\newenvironment{point}
   {\refstepcounter{point} \bigskip \noindent {\textbf{Reviewer~Point~\thepoint} } ---\ }
   {\par }

\newcommand{\shortpoint}[1]{\refstepcounter{point}  \bigskip \noindent 
	{\textbf{Reviewer~Point~\thepoint} } ---~#1\par }

\newenvironment{reply}
   {\medskip \noindent \begin{sf}\textbf{Reply}:\  }
   {\medskip \end{sf}}

\newcommand{\shortreply}[2][]{\medskip \noindent \begin{sf}\textbf{Reply}:\  #2
	\ifthenelse{\equal{#1}{}}{}{ \hfill \footnotesize (#1)}%
	\medskip \end{sf}}

    \newcommand{\rev}[1]{{\color{purple} #1}}

\begin{document}


\section*{Response to the reviewers}

We would like to thank the reviewer for the attention that our work has
received, and their very detailed review. We will address their concerns next


\subsection{Introduction}

\begin{point}
  Big data and Long data are now understandably a very popular exercise among economists, computer scientists and physicists. They have multiple analytical instruments at their service and can experiment with them using available historical data sets. The history of the Republic of Venice, especially the period 1400-1797 is an ideal playground: from an archival point of view, the creator (the State) was stable throughout the centuries; the archival series, except for damages due to fires and some dispersion following the 1797 French takeover of the city, remained intact; and lastly, Venetian society was rather stable to allow the constitution of data sets with data density uncommon in other early-modern States.

The present article by JJ Merelo, titled: Finding turning points in the history
of the Republic of Venice through statistical analysis of the time series of
noble marriages, is one of these experiments that try to analyze a phenomenon
over a long period, and establish turning points within it. This may be very
useful to historians who do not possess skills allowing them to deal with
complex calculations, equations, graphs etc., as it allows them another – and
perhaps a stimulating – point of view of the historical phenomenon they are
studying.  So, in this sense, I truly welcome the Author’s article, as it uses
historical big data to detect unseen or submerged social processes that may
elude the historian’s detection. I am a bit perplexed regarding the historical
analysis which tends to be considered of minor importance against the proposed
statistical experiment. The danger in this lack of balance between statistical
complexity and historical complexity is that the outcome may sometimes lead to
interesting statistical findings associated with irrelevant historical facts.
\end{point}

\begin{reply}
Thanks a lot for this accurate summary of our work. Following your advice (and
that of the rest of the reviwewers) we will try to make a more precise causal
relationship between statistical turning points and historical events.
We understand that this is a general remark, so no specific change is tied to
it. We will include other changes suggested in the rest of the section.
\end{reply}

\subsection{Major points and suggestions}

\begin{point}
  In this sense I think that slightly changing the title would contribute to
  mitigate some misunderstandings between the Author and the public of
  historians. When the Author writes: Finding turning points in the history of
  the Republic of Venice, any reader would assume that the concept of “turning
  point” would be discussed to delineate what he means by it. Instead, it is
  only at the end of the article that he relates to the matter: “At any rate,
  using multi-change point analysis allows a more natural division of the
  history of a certain state of region in periods, focusing on the effects of
  internal or external events (or combinations thereof), rather than on
  (possible) causes without demonstrated effects” (p. 13, lines 379-381). I
  wholly agree. This means a change of perspective: instead of defining turning
  points beforehand and then look for an explanation for their occurrence, the
  Author states that the analysis of the long phenomenon itself should be able
  to identify the turning points and proceed to search for correlations of
  events that occurred at that time.
\end{point}

\begin{reply}
  We have changed the title, as advised later in the same paragraph.
\end{reply}

\begin{point}
  Consequently, would it not be better to change the title into: Finding turning points in the time series of noble marriages during the time of the Republic of Venice? or: Detecting turning points by using multi-change point analysis of noble marriages during the time of the Republic of Venice? A new title would clear the table from any expectation to see a slow social phenomenon explaining, or explained by, an event which is a particular point in time that may or may not be the principal trigger for the slow social phenomenon.
\end{point}

\begin{reply}
  We have changed the title to {\em Detecting pivotal moments by using change point analysis of noble marriages during the time of the Republic of Venice}. Since we use single- and multi- change point analysis, we thought it was better to not refer to just one of the techniques used. On the other hand, we think it is important that we arrive at history from statistics, so we have changed {\em turning points} to {\em pivotal moments}, which also avoids repetition of the word {\em points} in the title.
\end{reply}

\begin{point}
  And this brings me to the second point and to the question whether a
  detection of what the Author calls “a turning point” through the
  statistical analysis of Venetian noble marriages can
  match/explain/be explained by an event. As an historian, I will
  perhaps have some difficulty to accept the “turning point” concept
  applied to marriage custom analysis for a simple reason: if one
  wishes to establish a relationship between an event or a series of
  events and the shift in social behavior he should take into
  consideration that a change in social behavior is a slow phenomenon
  whose manifestation will sometime occur a generation later.
\end{point}

\begin{reply}

  We have addressed this point by adding this text:
  \begin{quote}
    These pivotal moments or turning points might have occurred some time before the change point, which would then be a consequence of the pivotal moment, or will occur after, with the change in marriage trends creating a different moment; in general, social changes take hold very slowly and may take a generation or even more to actually appear in historical data. {\em A priori}, these turning points could be simply the consequence of change of customs related to marriage, but we are in general more interested in what caused those changes in customs and how they relate to specific internal or external events in the history of the Republic of Venice.
  \end{quote}

  In general, the importance of the nobility as ruling class, and what we are
  trying to find out in the paper, is how changes in matrimonial numbers or
  customs reflect history at large, rather than how this change is circumscribed
  to the time series of marriages itself. In general, that is going to be the
  case too. Demographic changes will be caused by historical events (war,
  famines) but will affect the number of marriages; economic expansion will
  affect it in a different way. Changes in class composition will also lead to
  changes in the concentration of noble marriages in specific families; and
  changes in class composition, in turn, are caused by other epochal events. In
  general, our interest is to find the causal chain that will cause the
  statistical change points, or what chain of events is triggered by a change
  point that eventually results in an epochal shift. But we understand that your
  main point is the inherent difficulty of discovering those historical events
  and establishing a causal relationship, since they might be hidden in an
  obscure decision taken thirty years earlier. What we will try, then is,
  following your advice, to reconsider or qualify the presented historical
  events along the lines of the following suggestions you have made.
\end{reply}

\begin{point}
  Looking at the six periods of “turning points” of marriage customs suggested
  by the Author, and proposing a more resilient analysis of slow social changes
  triggered by past events, things become clearer:
\end{point}


\begin{reply}
  Thanks for this detailed analysis. We will go in two different directions
  here. First, include the importance of the turning point as revealed by the
  statistical analysis, and then address your specific concerns. We have also
  included a table summarizing the changes observed at the pivotal years; this
  is now Table 2. We have also changed from an environment that has an itemized
  list to subsections, since the description of every period was starting to
  have an extension that was better served by dividing into paragraphs.
\end{reply}

\begin{point}
  {\bf First period – 1399-1435} – characterized by middle values of entropy,
  low number of marriages, and a medium number of non-patrician marriages -
  beside the point that for this period we do not have a complete data set, we
  should bear in mind that in 1422 the Great Council further clarifies the
  status of a marriageable woman able to allow her son inherit his father’s
  noble status. I recommend citing here Chojnacki 1994 and would consider a
  reformulation of p. 13, lines 336-240: “which required every candidate to new
  member of the Consiglio to be presented by his father. Although this might
  seem like a minor requirement, this measure was accompanied by a series of
  sumptuary laws, but more importantly by a bureaucratization of all processes
  related to marriages, including dowry contracts”, taking into consideration of
  the 1422 law (which was a turning point, but its impact was a slow one).
\end{point}

\begin{reply}
We have rewritten a big part of this point, focusing on events occurring {\em
  during} this period which caused a turning point at the end. We have inserted
the reference, but also added a reference to \cite{10.2307/202860}, which makes
a very detailed analysis of the amounts given as dowries. The periods under
study in that paper slightly overlap this one, and are characterized by dowries
that were so high that most families could afford to marry off only one
daughter. This might explain the low number of marriages, but also the turning
point at the and of this period, possibly caused by the actual enforcement of
sumptuary laws that limited the amount that had to be paid as such.

In general, although these measures did have a long-term social and cultural
impact, they are certainly not watershed events, and its influence was largely
circumscribed to the sphere of marriage itself. This explains the fact that,
among all turning points, this is the one with the lowest score.
\end{reply}

\begin{point}

  {\bf Second period – 1435-1525} - characterized by “an increment in the number of
  marriages, no change in entropy, but mainly resulted in a decrease of the
  marriages that included a non-noble partner” (p. 13, lines 342-343). Here the
  Author tries to explain the phenomenon by all those laws he mentioned before,
  mainly sumptuary laws (which I fail to understand the their relationship to
  marriage fluctuations)

\end{point}

\begin{reply}

  Allow us to split this point into different items, so that we can properly
  address it. The relationship has been justified in the previous reply by a
  reference to \cite{10.2307/202860}. Limiting the amount that had to be paid as
  dowry (and effectively enforcing that limit via notarization of contracts)
  allowed, for some families, to increment the amount of daughters that could be
  married, instead of limiting it to a single one.
\end{reply}

\begin{point}
  [...] And to the fact that “every candidate to new member of the Consiglio […]
  be presented by his father” (p. 13, lines 336-337) (which, formulated as such,
  is meaningless unless the Author alludes to the introduction in 1506 of the
  register of noble births and in 1526 of the register of noble marriages, but
  more on this in the third period).
\end{point}

\begin{reply}
This has now been clarified by referencing \cite{second:serrata}, including this
as a footnote to that sentence:

\begin{quote}
  This apparently formal act has in fact a deeper meaning in the clarification
  of the \emph{identity} of the nobles, as well as its stated purpose: avoiding
  people entering the nobility (through their co-optation into the Maggior
  Consiglio) without having the required requisites. This law was approved in
  1414 \cite{second:serrata}.
\end{quote}

According to several papers by Chojnacki, including \cite{chojnacki00}, the
construction of the identity of the noble class in Venice extended across many
centuries, and was continuously evolving. What you mention is mentioned in that
paper as the {\em third} Serrata (you probably know this already), and would
certainly have an influence in the change point that happened a few years
later. However, this is not what I implied in this sentence, which I expect is
now clarified.
\end{reply}

\begin{point}

  In fact, this period is hard to explain by an introduction of a specific
  law. One might surmise that the repercussion of the 1422 law slowly but surely
  had an impact on the decrease in marriages of a non-noble partner, but how do
  we explain the overall increase in marriages? Well, if we take into
  consideration the overtaking of the Mainland (Terraferma) reaching its peak
  during the reign of doge Francesco Foscari (from 1423 to 1457) and the wealth
  generated by the conquest, it follows that the Venetian nobles tend to
  increase the number of offspring per family, and consequently celebrate more
  marriages.
\end{point}

\begin{reply}
This is indeed a very good point which I expect you don't mind if we incorporate
into the paper. We cannot certainly reference a ``personal communication by
Reviewer \# 2'', so we have included \cite{law:mainland} as a source for this,
as well as acknowledgement of your help in the corresponding section. This
period now includes this statement:\begin{quote}
  This period starts soon after the beginning of the tenure of doge Francesco
  Foscari \cite{law:mainland}, which was characterized by an expansion into {\em
    terraferma}, that is, the mainland, that included many cities in the
  continent, as well as a progressive change of nobility from merchants to
  producers. This increase in wealth and consequent demographic expansion will
  explain the raise in the number of marriages; on the other hand, restrictions
  to marriage to non-nobles that included, by the end of the period, the so
  called {\em third Serrata} \cite{chojnacki00} accounts for the decrease in the
  number of marriages with non-patricians as well as the maintenance of the
  entropy.
\end{quote}
\end{reply}

\begin{point}
  Perhaps confronting the incredible graphic similarity in trend of marriages,
  especially in the 16th century (fig. 4 on p. 6) with the evolution in the
  number of male patricians (Raines (2013), figures 11 on p. 138 and 12 on
  p. 139) may contribute to explain the phenomenon of increase in the number of
  marriages in the second period, to continue up to a point in the third one).
\end{point}

\begin{reply}
  Again, we need to thank you for this very helpful suggestion. We have included
  this new text for the explanation of this period \begin{quote}
This demographic increment is also reflected in \cite{raines2013rameau}; the end
of this period is characterized by a peak in the number of nobles with voting
rights ({\em op. cit.}, Graphique VII  ). We are then talking about a number of
legislative, economical and political factors that, combined, have contributed
to this pivotal point by the end of the period, which matches the peak number of
patrician voters.
\end{quote}
\end{reply}

\begin{point}
  {\bf Third period – 1525-1593} - characterized by “the lowest average percentage of
  non-patrician marriages, which was accompanied also by a low entropy” and by
  the number of marriages which reaches its peak in 1552 (p. 13, lines 344-361).
\end{point}

\begin{reply}
We will again split the different points in this period to address them
properly. We will also add, as indicated at the beginning, the importance of the
change point at the end of the period.
\end{reply}

\begin{point}
First, I think that referring always to what Chojnacki calls “the second – and
third Serrata” is confusing to those who are not historians of Venice. Why not
refer to the 1414 Barbarella register as a substitute to the second Serrata
(Chojnacki 1986) and to the 1506 opening of the birth register and the 1526 of
the marriage one as a substitute of the third one? These are the very events
that should be mentioned.
\end{point}

\begin{reply}
  We have done so, referring to the papers where Chojnacki introduced the
  terms. However, we have kept the reference to the second and third Serrata as
  singular, legislative events already identified as such that could explain statistically found change
  points. We have also included a reference to \cite{chojnacki86} as
  mentioned. This paper clearly indicates that, despite the Barbarella existing
  for some time before, the keeping of registries was enacted in 1414.
\end{reply}


\subsection{Minor points}

\begin{point}
  p. 1, lines 32-36 – the doge is not the President of the republic. His rank is Prince, but he is also “first among equals”. I would reformulate as follows: “its state was governed by a representative figure, with a rank of prince, called doge, elected for life, initially by popular acclamation, then from the 12th century by all male nobles, members of the Great Council, the sovereign body of the Republic Maranini (1927); Ruggiero (1979)”.
	\label{pt:m:1}
      \end{point}

      \begin{reply}
Fixed as suggested. Thanks for the suggestion.
\end{reply}

\begin{point}
   p. 2, lines 81-82: “and then showed that the time spent in doges decreased
   extensively after that date”. I suppose the Author means: “…that the time
   doges remained in office decreased….”
   	\label{pt:m:2}
 \end{point}

 \begin{reply}
   Done as suggested. Thanks again.
 \end{reply}

\begin{point}
p. 4, figure 1 – “blue shows marriages within the same family” – I think the Author should explain that the extended family has a number of branches (sometimes 2-3, some others up to 30 or even more). When one branch has only a female member, the usual custom is to get her married to a male from a close branch in order not to lose the heritage, nor the potential votes of family members. It might be wise to explain this somewhere in the article.
   	\label{pt:m:3}
 \end{point}

 \begin{reply}
   This was a caption to a Figure. It has been extended and explained further in
   the text, citing Raines 2013. The mention to why some of those marriages took
   place when a there was a single female in the {\em ramo} was probably out of
   scope for the paper, and I could not find a precise source for it, so I left
   it out.
 \end{reply}

 \begin{point}
p. 6 lines 183-185 – “Taking into account that marriage can be considered a market Becker (1973), it is quite clear that the number of marriages will depend on the supply of available (noble) males” – I would suggest looking also at Munno, Derosas (2015) for the Venetian noble marriage market and its strategies.
   	\label{pt:m:4}
 \end{point}

 \begin{reply}
That reference has been included in the bibliography and cited right behihd
Becker. It certainly is a good presentation of the marriage market, focused on
the last years of the Republic. We have also explained what kind of ``goods and
services'' would be exchanged in that market: dowries and, well, persons for
political influence and commercial alliances.
 \end{reply}

\bibliographystyle{unsrt}
\bibliography{marriage,venice}
\end{document}
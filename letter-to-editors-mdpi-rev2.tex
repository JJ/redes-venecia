% LaTeX rebuttal letter example. 
% 
% Copyright 2019 Friedemann Zenke, zenkelab.org
%
% Based on examples by Dirk Eddelbuettel, Fran and others from 
% https://tex.stackexchange.com/questions/2317/latex-style-or-macro-for-detailed-response-to-referee-report
% 
% Licensed under cc by-sa 3.0 with attribution required.

\documentclass[11pt]{article}
\usepackage[utf8]{inputenc}
\usepackage{fullpage}
\usepackage{xcolor}

% import Eq and Section references from the main manuscript where needed
% \usepackage{xr}
% \externaldocument{manuscript}

% package needed for optional arguments
\usepackage{xifthen}
% define counters for reviewers and their points
\newcounter{reviewer}
\setcounter{reviewer}{0}
\newcounter{point}[reviewer]
\setcounter{point}{0}

% This refines the format of how the reviewer/point reference will appear.
\renewcommand{\thepoint}{P\,\thereviewer.\arabic{point}} 

% command declarations for reviewer points and our responses
\newcommand{\reviewersection}{\stepcounter{reviewer} \bigskip \hrule
                  \section*{Reviewer \thereviewer}}

\newenvironment{point}
   {\refstepcounter{point} \bigskip \noindent {\textbf{Reviewer~Point~\thepoint} } ---\ }
   {\par }

\newcommand{\shortpoint}[1]{\refstepcounter{point}  \bigskip \noindent 
	{\textbf{Reviewer~Point~\thepoint} } ---~#1\par }

\newenvironment{reply}
   {\medskip \noindent \begin{sf}\textbf{Reply}:\  }
   {\medskip \end{sf}}

\newcommand{\shortreply}[2][]{\medskip \noindent \begin{sf}\textbf{Reply}:\  #2
	\ifthenelse{\equal{#1}{}}{}{ \hfill \footnotesize (#1)}%
	\medskip \end{sf}}

    \newcommand{\rev}[1]{{\color{purple} #1}}

\begin{document}


\section*{Response to the reviewers}

We would like to thank the reviewer for the attention that our work has
received, and their very detailed review. We will address their concerns next

\reviewersection

\subsection{Major points and suggestions}

\begin{point}
  And this brings me to the second point and to the question whether a
  detection of what the Author calls “a turning point” through the
  statistical analysis of Venetian noble marriages can
  match/explain/be explained by an event. As an historian, I will
  perhaps have some difficulty to accept the “turning point” concept
  applied to marriage custom analysis for a simple reason: if one
  wishes to establish a relationship between an event or a series of
  events and the shift in social behavior he should take into
  consideration that a change in social behavior is a slow phenomenon
  whose manifestation will sometime occur a generation later.
\end{point}

\begin{reply}

  We have addressed this point by adding this text:
  \begin{quote}
    These pivotal moments or turning points might have occurred some time before the change point, which would then be a consequence of the pivotal moment, or will occur after, with the change in marriage trends creating a different moment; in general, social changes take hold very slowly and may take a generation or even more to actually appear in historical data. {\em A priori}, these turning points could be simply the consequence of change of customs related to marriage, but we are in general more interested in what caused those changes in customs and how they relate to specific internal or external events in the history of the Republic of Venice.
  \end{quote}

  In general, the importance of the nobility as ruling class, and what we are trying to find out in the paper, is how changes in matrimonial numbers or customs reflect history at large, rather than how this change is circumscribed to the time series of marriages itself. In general, that is going to be the case too. Demographic changes will be caused by historical events (war, famines) but will affect the number of marriages; economic expansion will affect it in a different way. Changes in class composition will also lead to changes in the concentration of noble marriages in specific families; and changes in class composition, in turn, are caused by other epochal events. In general, our interest is to find the causal chain that will cause the statistical change points, or what chain of events is triggered by a change point that eventually results in an epochal shift.
\end{reply}

\begin{point}
  Consequently, would it not be better to change the title into: Finding turning points in the time series of noble marriages during the time of the Republic of Venice? or: Detecting turning points by using multi-change point analysis of noble marriages during the time of the Republic of Venice? A new title would clear the table from any expectation to see a slow social phenomenon explaining, or explained by, an event which is a particular point in time that may or may not be the principal trigger for the slow social phenomenon.
\end{point}

\begin{reply}
  We have changed the title to {\em Detecting pivotal moments by using change point analysis of noble marriages during the time of the Republic of Venice}. Since we use single- and multi- change point analysis, we thought it was better to not refer to just one of the techniques used. On the other hand, we think it is important that we arrive at history from statistics, so we have changed {\em turning points} to {\em pivotal moments}, which also avoids repetition of the word {\em points} in the title.
\end{reply}

\subsection{Minor points}

\begin{point}
  p. 1, lines 32-36 – the doge is not the President of the republic. His rank is Prince, but he is also “first among equals”. I would reformulate as follows: “its state was governed by a representative figure, with a rank of prince, called doge, elected for life, initially by popular acclamation, then from the 12th century by all male nobles, members of the Great Council, the sovereign body of the Republic Maranini (1927); Ruggiero (1979)”.
	\label{pt:m:1}
      \end{point}

      \begin{reply}
Fixed as suggested. Thanks for the suggestion.
\end{reply}

\begin{point}
   p. 2, lines 81-82: “and then showed that the time spent in doges decreased
   extensively after that date”. I suppose the Author means: “…that the time
   doges remained in office decreased….”
   	\label{pt:m:2}
 \end{point}

 \begin{reply}
   Done as suggested. Thanks again.
 \end{reply}

\begin{point}
p. 4, figure 1 – “blue shows marriages within the same family” – I think the Author should explain that the extended family has a number of branches (sometimes 2-3, some others up to 30 or even more). When one branch has only a female member, the usual custom is to get her married to a male from a close branch in order not to lose the heritage, nor the potential votes of family members. It might be wise to explain this somewhere in the article.
   	\label{pt:m:3}
 \end{point}

 \begin{reply}
   This was a caption to a Figure. It has been extended and explained further in
   the text, citing Raines 2013. The mention to why some of those marriages took
   place when a there was a single female in the {\em ramo} was probably out of
   scope for the paper, and I could not find a precise source for it, so I left
   it out.
 \end{reply}

 \begin{point}
p. 6 lines 183-185 – “Taking into account that marriage can be considered a market Becker (1973), it is quite clear that the number of marriages will depend on the supply of available (noble) males” – I would suggest looking also at Munno, Derosas (2015) for the Venetian noble marriage market and its strategies.
   	\label{pt:m:4}
 \end{point}

 \begin{reply}
That reference has been included in the bibliography and cited right behihd
Becker. It certainly is a good presentation of the marriage market, focused on
the last years of the Republic.
 \end{reply}

\bibliographystyle{unsrt}
\bibliography{marriage}
\end{document}
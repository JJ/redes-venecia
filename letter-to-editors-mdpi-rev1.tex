% LaTeX rebuttal letter example. 
% 
% Copyright 2019 Friedemann Zenke, zenkelab.org
%
% Based on examples by Dirk Eddelbuettel, Fran and others from 
% https://tex.stackexchange.com/questions/2317/latex-style-or-macro-for-detailed-response-to-referee-report
% 
% Licensed under cc by-sa 3.0 with attribution required.

\documentclass[11pt]{article}
\usepackage[utf8]{inputenc}
\usepackage{fullpage}
\usepackage{xcolor}

% import Eq and Section references from the main manuscript where needed
% \usepackage{xr}
% \externaldocument{manuscript}

% package needed for optional arguments
\usepackage{xifthen}
% define counters for reviewers and their points
\newcounter{reviewer}
\setcounter{reviewer}{0}
\newcounter{point}[reviewer]
\setcounter{point}{0}

% This refines the format of how the reviewer/point reference will appear.
\renewcommand{\thepoint}{P\,\thereviewer.\arabic{point}} 

% command declarations for reviewer points and our responses
\newcommand{\reviewersection}{\stepcounter{reviewer} \bigskip \hrule
                  \section*{Reviewer \thereviewer}}

\newenvironment{point}
   {\refstepcounter{point} \bigskip \noindent {\textbf{Reviewer~Point~\thepoint} } ---\ }
   {\par }

\newcommand{\shortpoint}[1]{\refstepcounter{point}  \bigskip \noindent 
	{\textbf{Reviewer~Point~\thepoint} } ---~#1\par }

\newenvironment{reply}
   {\medskip \noindent \begin{sf}\textbf{Reply}:\  }
   {\medskip \end{sf}}

\newcommand{\shortreply}[2][]{\medskip \noindent \begin{sf}\textbf{Reply}:\  #2
	\ifthenelse{\equal{#1}{}}{}{ \hfill \footnotesize (#1)}%
	\medskip \end{sf}}

    \newcommand{\rev}[1]{{\color{purple} #1}}

\begin{document}


\section*{Response to the reviewers}

We would like to thank the reviewer for the attention that our work has received. We will address their concerns next

\reviewersection

\begin{point}
  Patricia Brown's book, The Venetian Bride, must be read and included in this paper's discussion.
	\label{pt:1:1}
      \end{point}

      \begin{reply}
The paper has been included in the bibliography. It is indeed central to the
main point of our paper, which is an analysis of the time series of
marriages. Certainly, how these marriages took place, were arranged and marriage
strategies, as discussed in Dr. Brown's book focusing on specific families and
lineages, should have been discussed. It now includes this text:

\rev{As \cite{brown2021venetian} states:
\begin{quote}
... Brides were the creation of male agency and mediation.
\end{quote}
And
\begin{quote}
Marriages were, as always, favoured ways to create new alliances and cement old ones.
\end{quote}
}

And, later on, as a footnote:

\rev{As mentioned in \cite{brown2021venetian}, dowry payments involved elaborate installment plans that included cash as well as, sometimes, payment in kind}

        \end{reply}

\bibliographystyle{unsrt}
\bibliography{marriage}
\end{document}
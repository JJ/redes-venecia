% LaTeX rebuttal letter example. 
% 
% Copyright 2019 Friedemann Zenke, zenkelab.org
%
% Based on examples by Dirk Eddelbuettel, Fran and others from 
% https://tex.stackexchange.com/questions/2317/latex-style-or-macro-for-detailed-response-to-referee-report
% 
% Licensed under cc by-sa 3.0 with attribution required.

\documentclass[11pt]{article}
\usepackage[utf8]{inputenc}
\usepackage{fullpage}
\usepackage{xcolor}
\usepackage{xr}
\externaldocument{change-point-marriages-venice}

% import Eq and Section references from the main manuscript where needed
% \usepackage{xr}
% \externaldocument{manuscript}

% package needed for optional arguments
\usepackage{xifthen}
% define counters for reviewers and their points
\newcounter{reviewer}
\setcounter{reviewer}{5}
\newcounter{point}[reviewer]
\setcounter{point}{0}

% This refines the format of how the reviewer/point reference will appear.
\renewcommand{\thepoint}{P\,\thereviewer.\arabic{point}} 

% command declarations for reviewer points and our responses
\newcommand{\reviewersection}{\stepcounter{reviewer} \bigskip \hrule
                  \section*{Reviewer \thereviewer}}

\newenvironment{point}
   {\refstepcounter{point} \bigskip \noindent {\textbf{Reviewer~Point~\thepoint} } ---\ }
   {\par }

\newcommand{\shortpoint}[1]{\refstepcounter{point}  \bigskip \noindent 
	{\textbf{Reviewer~Point~\thepoint} } ---~#1\par }

\newenvironment{reply}
   {\medskip \noindent \begin{sf}\textbf{Reply}:\  }
   {\medskip \end{sf}}

\newcommand{\shortreply}[2][]{\medskip \noindent \begin{sf}\textbf{Reply}:\  #2
	\ifthenelse{\equal{#1}{}}{}{ \hfill \footnotesize (#1)}%
	\medskip \end{sf}}

\newcommand{\rev}[1]{{\color{purple} #1}}

\begin{document}


\section*{Response to the reviewers}

We would like to thank the editor and reviewers for the attention that our work
has received. We will address their concerns next.

\begin{point}
  Dear authors, the paper is indeed a significant contribution to the
  large-scale analysis of marriage trends, one of the elements that emerge from
  archival sources and potentially a historically significant marker. The
  reading of the dataset, that you have correctly filtered and purged of
  irrelevant, or at least not homogeneous, elements seems to me
  convincing. Especially the methodology used to find the change points and the
  application of entropy to the evolution of the dataset.
\end{point}

\begin{reply}
  Thanks. In this new version of the paper, an additional changepoint analysis
  method has been tested on the entropy time series. Although it yields a change
  point that is not exactly the same as the ones we had found, it occurs in the
  context of the same general events (late 17th century) that have been
  identified as the most important one in this marriage series.
\end{reply}

\begin{point}
However, it seems to me methodologically a bit weak that you have not tried to
introduce as a variable the calculation of the global population, which has also
been done for Venice (several studies have discussed it and simulated it) and
which perhaps would have been correct to include. Particularly if you are
talking about a drastic decrease, as well as an increasing trend, up to the
years "1570," a normalization on the assumed numbers of the whole population
seems to me to be due. For there could be an increase in patrician marriages
that in fact confirms a general upward trend in the population (as indeed it
seems to me until almost the end of the sixteenth century) and instead a descent
due to an overall reduction in the size of the population. In this comparison
any element of discontinuity, e.g., matrimony increases but population
decreases, would gain even more significance, hence the methodological need to
try to create a graph with normalization of marriages based on population
quantity.
\end{point}

\begin{reply}
  The reviewer is correct in pointing out that general demography will have an
  influence in a time series that is, first and foremost, about
  population. There is an issue here that is impossible to surmount: there is
  simply not a dataset with the same granularity than the one we have for
  marriages. This is simply not a lack of published datasets, it is simply that,
  since registers were associated to citizenship, the {\em popolani} or popular
  classes did not have any kind of registers, so there is no such thing as a
  year (or even ten-year) time series of Venice population. Some authors,
  however, have either estimated or taken total population from some sources
  \cite{davis1962decline} (although at irregular intervals spanning sometimes
  more than 50 years) or computed, in this case using legal registries, the
  total noble population \cite{raines2013rameau}.

  This brings us to the second point. When trying to find the structural changes
  that have cause a changepoint in the number of marriages (and other measures
  related to it) it is essential to identify if this is due to a change in the
  total population (and nobles as a part of it), in the noble population (which
  might have increased or shrunk with respect to the total population) or in the
  number of nobles that actually marry from all the available population. And it
  is important to find out the distinction, because in the two first cases we
  could be talking about a far-reaching cause that will affect the history of
  Venice for the rest of its history, while in the last case it would be a
  change in social mores whose impact, most probably, would be reduced to the
  noble class itself, its definition and its insertion into the population.

  There are different ways to approach this; one of the reviewers has suggested
  using Bayesian changepoint analysis, which would be able to find which models
  of change would statistically best explain the situation after the
  changepoint. This is a valid approach, but it is beyond the scope of our
  skills or, for that matter, a journal focused on history. Another reviewer has
  suggested to delve into the historical events, be it legal, natural, or
  artificial, and has even suggested some events that might explain the peak in
  marriages as well as the different changepoints found by the statistical
  analysis. We have taken the last route, which offers a less steep onramp for
  historians with some statistical knowledge; that way, we have greatly expanded
  section 5 trying to explain, via historical pivotal events, changes in the
  number of marriages as well as in the composition of those marriages. Besides
  the textual additions, we have added Table 4 that lists probable causes of
  changes, including demographic when they have been found to be so.
\end{reply}

\begin{point}
  On the other hand, as far as trying to align with already established
  historical knowledge to understand the nature of change points, I find that
  ignoring the two great plagues of 1570(circa) and 1630 (circa) is a bit of a
  major oversight. Undoubtedly wars affect the "availability" of male
  population, but the plagues, which reduced, in the two cases, the overall
  population by at least one-third, seem to me to be very important to
  consider. In particular, I would like to point out that if there was a very
  high infant mortality rate (due to the plague) right around 1575-1577 and then
  1630, this causes a reduction in the generation of births in those years and
  that strongly affects the availability of marriage-age population at that time
  in history and for the next 25 years (one generation). This is to say that
  although the attempt at historical analysis is there, I think this element
  needs to find a place to be discussed. Perhaps it can be summarized in a
  paragraph that I would put in the "detecting change points" section and then
  reemphasize on p. 13 in the "periods."
\end{point}

\begin{reply}
As indicated above, historical causes, including plagues, are now discussed not
only in Section 4, which was devoted to explain the peak in population, but also
in the different periods. The 1556 plague was, for instance, already mentioned
in that section. The 1570 plague is now mentioned in the third period (that ends
in 1593), and the 1630 plague
is now mentioned in a footnote citing \cite{raines2003cooptazione}.

A different plague, that of gonorrhea, was mentioned by \cite{davis1962decline}
as the cause of the decline in {\em fertility} of (mainly) the noble class
(certainly a consequence of {\em uomi} not being so {\em nobili}, if you allow
the pun). As indicated above, this is a historically proved event that is, in
this version, identified as a cause of the structural shift revealed by the 1593
changepoint.
\end{reply}


\bibliographystyle{unsrt}
\bibliography{marriage,venice,change-point,history}
\end{document}
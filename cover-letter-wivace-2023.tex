% Cover letter boilerplate in LaTeX

\documentclass[11pt]{article}
\usepackage{fullpage}
\usepackage{hyperref}
\usepackage{url}

\begin{document}

Dear sirs:\\

This is a cover letter for the submission of the final version of the paper "Self-loops in social networks: behavior of eigenvector centrality", which was presented in the WIVACE 2023 conference in Venice. This is a list of the changes made to the original submission.\begin{itemize}
\item The original paper was an extended abstract, 4 pages long. This final version is a full paper, 14 pages.
\item This paper has been divided in sections and extended with a section detailing the relevant state of the art.
\item The comments made by the reviewers have also been addressed:\begin{itemize}
  \item \textbf{Reviewer 1: the paper is limited to a single network example}: the paper now includes two examples, and in the network that had been used before, the Venice matrimonial networks, analysis have been made to (significant) slices of the network. This new network, a freight traffic network, has characteristics that are totally different from the one included in the extended abstract, making conclusions more solid.
  \item \textbf{Reviewer 2: Consider a scatterplot of the ranking of families, not only the values}: This is the new Figure 3, which is also commented in the text; it has enabled us to reach the conclusion that the changes in ranking occur mostly in one direction, lowering the rank once self-loops are considered. We are grateful to the reviewer for this suggestion.
  \item \textbf{Reviewer 2: Correlation between measures with and without self-loops}: The correlation for measures with and without self-loops and the changes in ranking are now more extensively commented in the text for both networks examined.
  \item \textbf{Reviewer 2: How self-loops influences the results in the original paper}:  what we have done, following Puga and Treffler paper, is to compare the models for evolution of EV centrality of the matrimonial network in the XIV and XV century; we have found that using self-loops is the best fit for the data. Again, we are grateful to the reviewer for this suggestion
  \end{itemize}
  \end{itemize}
  
Hope all these changes make it fit for publication. Yours\\


\centerline{M. C. Molinari \& J. J. Merelo}

\end{document}